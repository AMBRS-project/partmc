
\documentclass[12pt]{amsart}
\usepackage{geometry} % see geometry.pdf on how to lay out the page. There's lots.
\geometry{a4paper} % or letter or a5paper or ... etc
% \geometry{landscape} % rotated page geometry

% See the ``Article customise'' template for come common customisations

\title{Review of droplet condensational growth equations}
\author{Joseph Ching}
\date{24 Nov 2008} % delete this line to display the current date

%%% BEGIN DOCUMENT
\begin{document}

\maketitle
\tableofcontents

This review contains the derivation of droplet condensation growth equations based on Seinfeld and Pandis, Majeed and Wexler, and James Kelly. All the derivations involve the uses of Ideal gas law, Kelvin equation, Clausius-Clapeyron equation, the conservation of energy and the diffusion equation.

\section{Definition of symbols}

\begin{itemize}
\item $c_{w}^{eq}$ : equilibrium/saturated vapor pressure
\item $c_{w, \infty}$ : environmental vapor pressure (vapor pressure far away from the droplet surface)
\item $D_{p}$: diameter of droplet 
\item $D_{v}$: diffusion coefficient (not corrected for non-continuum effect )
\item $d_u$ : diameter of insoluble material
\item $k^{'}_{a}$ : thermal conductivity of air (corrected for non-continuum effect )
\item $\Delta H_{v}$: latent heat of condensation
\item $\rho_{w}$ : density of water
\item $m$ : mass of droplet
\item $m_d$ : mass of droplet
\item $m_p$ : mass of dry particle
\item $m_s$ : mass of solute (soluble part of aerosol)
\item $M_{s}$ : molecular weight of salt
\item $M_{w}$ : molecular weight of water
\item $P^{o}$ : environmental saturated water vapor pressure 
\item $P_{w,\infty} $: environmental actual water vapor pressure
\item $P_{w}(D_{p}, T_{a}) $ : saturated vapor pressure at droplet surface of diameter $ D_{p}$ at temperature $T_{a}$  
\item $R$ : universal gas constant
\item $\textrm{RH}$ : relative humidity
\item $S_{v, \infty}$ : Supersaturation  
\item $T_{a}$: temperature at the droplet surface
\item $T_{\infty}$ : environmental temperature (temp. far away from the droplet surface)
\item $\nu$ : number of ions produced per molecule after complete dissociation of salt 
\end{itemize}

 
\section{Seinfeld and Pandis (SP) }

\subsection{derivation of the condensational growth equation of a droplet}


The derivation is contained in chapter 15 (pp. 801- 803) of Senfeld and Pandis. The following derivations just show the mathematical derivation in more detail.

The derivation starts from the equation of diffusion of mass which is given as
\begin{equation}\label{eqn:1}
\frac{dm}{dt}=2 \pi D_{p} D_{v} (c_{w, \infty} - c_{w}^{eq}).  
\end{equation}

To express the growth equation in terms of diameter rate of change, instead of mass rate of change,

\begin{equation}\label{eqn:2}
\frac{dm}{dt}=\frac{1}{2} \pi \rho_{w} D_{p}^{2} \frac{dD_{p}}{dt}
\end{equation}

and them put eqn \ref{eqn:2} into the the eqn \ref{eqn:1}, gives

\begin{equation}\label{eqn:3}
\frac{1}{2} \pi \rho_{w} D_{p}^{2} \frac{dD_{p}}{dt}= 2 \pi D_{p} D_{v} (c_{w, \infty} - c_{w}^{eq}),
\end{equation}

and after rearranging we get, 
\begin{equation}\label{eqn:4}
D_{p}\frac{dD_{p}}{dt}=4 \frac{D_{v}} {\rho_{w}} (c_{w, \infty} - c_{w}^{eq}).
\end{equation}

By ideal gas law, the eqn \ref{eqn:4} can be expressed as,

\begin{equation}\label{eqn:5}
D_{p}\frac{dD_{p}}{dt}=4 \frac{D_{v}} {\rho_{w}} \frac{M_{w}}{R} (\frac{P_{w, \infty}}{T_{\infty}} - \frac{P_{w}(D_{p}, T_{a})}{T_{a}}).
\end{equation}

To have the term environmental relative humidity or supersaturation entered into eqn \ref{eqn:5} as folows,  
\begin{equation}\label{eqn:6}
D_{p}\frac{dD_{p}}{dt}=4  \frac{D_{v}} {\rho_{w}} \frac{M_{w}}{R} \frac{P^{o}(T_{\infty})} {T_{\infty}} (\frac{P_{w, \infty}}{P^{o}(T_{\infty})} - \frac{P_{w}(D_{p}, T_{a}) T_{\infty}}{P^{o}(T_{\infty}) T_{a}}).
\end{equation}

To relate the temperature of the environment, $T_{\infty}$ to that on the droplet surface, $T_{a}$, we need the argument of the conservation of energy, to derive the energy balance,

\begin{equation}\label{eqn:7}
2 \pi D_{p} k^{'}_{a} (T_{\infty}-T_{a})= -\Delta H_{v}(\frac{dm}{dt}).
\end{equation}

and further express \ref{eqn:7} in terms of diameter growth, 

\begin{equation}
2 \pi D_{p} k^{'}_{a} (T_{\infty}-T_{a})= -\Delta H_{v} \frac{1}{2} \pi \rho_{w} D_{p}^{2} \frac{dD_{p}}{dt},
\end{equation}

and after rearranging we get, 
\begin{equation}
T_{a}= T_{\infty} + \frac{\Delta H_{v}  \rho_{w}}{4 k^{'}_{a}} D_{p} \frac{dD_{p}}{dt},
\end{equation}

and write as 
\begin{equation}
T_{a}=T_{\infty}(1+\delta),
\end{equation}

where 

\begin{equation}\label{eqn:SPdelta}
\delta= \frac{\Delta H_{v}  \rho_{w}}{4 k^{'}_{a}T_{\infty}} D_{p} \frac{dD_{p}}{dt}.
\end{equation}

By substituting $T_{a}$ in the eqn \ref{eqn:6} in terms of $T_{\infty}$, we get 


\begin{equation}\label{eqn:12}
D_{p}\frac{dD_{p}}{dt}=4 \frac{D_{v}} {\rho_{w}} \frac{M_{w}}{R} \frac{P^{o}(T_{\infty})} {T_{\infty}} (S_{v, \infty} - \frac{P_{w}(D_{p}, T_{a})}{P^{o}(T_{\infty})} \frac{1}{1+\delta}).
\end{equation}

To further calculate the ratio of $\frac{P_{w}(D_{p}, T_{a})}{P^{o}(T_{\infty})}$, we need to use Kelvin 's equation and Clausius-Clapeyron equation, which describe the effect of curvature on the equilibrium water vapor pressure as function of droplet size and the equilibrium water vapor pressure as function of temperature respectively. As a remark, the term equilibrium water vapor pressure and saturated vapor pressure are used interchangeably in many text.

Before invoking the two equations, we need to write eqn \ref{eqn:12} as follows,
\begin{equation}\label{eqn:SPgrowth}
D_{p}\frac{dD_{p}}{dt}=4 \frac{D_{v}} {\rho_{w}} \frac{M_{w}}{R} \frac{P^{o}(T_{\infty})} {T_{\infty}} (S_{v, \infty} - \frac{P_{w}(D_{p}, T_{a})}{P^{o}(T_{a})} \frac{P^{o}(T_{a})}{P^{o}(T_{\infty})}\frac{1}{1+\delta}).
\end{equation}

By Kelvin equation, 

\begin{equation}\label{eqn:kelvin}
\frac{P_{w}(D_{p}, T_{a})}{P^{o}(T_{a})}=\exp( \frac{4 M_{w}\sigma_{w}}{R T_{a} \rho_w}\frac{1}{D_{p}}-\frac{6 n_{s} M_{w}}{\pi \rho_{w}} \frac{1}{D_{p}^{3}-d_{u}^{3}}).
\end{equation}

By Clausius and Clapeyron equation,

\begin{equation}
\frac{P^{o}(T_{a})}{P^{o}(T_{\infty})}=\exp (\frac{\Delta H_{v} M_{w}}{R} (\frac{T_{a}-T_{\infty}}{T_{a} T_{\infty}})).
\end{equation}

which can be further simplified into 

\begin{equation}\label{eqn:CC}
\frac{P^{o}(T_{a})}{P^{o}(T_{\infty})}=\exp (\frac{\Delta H_{v} M_{w}}{R T_{\infty}} \frac{\delta}{1+\delta}).
\end{equation}

Substituting the above two equations into eqn \ref{eqn:SPgrowth}, we get

\begin{equation}\label{eqn:SPfinal}
D_{p}\frac{dD_{p}}{dt}=4 \frac{D_{v}} {\rho_{w}} \frac{M_{w}}{R} \frac{P^{o}(T_{\infty})} {T_{\infty}} (S_{v, \infty} - \frac{1}{1+\delta} \exp(\frac{4 M_{w}\sigma_{w}}{R T_{a} \rho_w}\frac{1}{D_{p}}-\frac{6 n_{s} M_{w}}{\pi \rho_{w}} \frac{1}{D_{p}^{3}-d_{u}^{3}}+\frac{\Delta H_{v} M_{w}}{R T_{\infty}} \frac{\delta}{1+\delta}).
\end{equation}

The eqn \ref{eqn:SPfinal} agrees with the equation (15.73). According to S\&N, with the approximation of $\delta \ll 1$, equation (15.74) could be obtained from equation (15.73).  However, the author cannot reach equation (15.74).

Two more points to note are, first the temperature in the denominator of the term  $\frac{4 M_{w}\sigma_{w}}{R T_{a} \rho_w}\frac{1}{D_{p}}$ is $T_{a}$, the temperature of the droplet. In S\&P equation (15.73), the variable $T$ is not explicitly indicated as droplet or environmental temperature. The second point to note is that the diffusion coefficient in S\&P, denoted by $D_{v}^{'}$  is the one corrected for non-continuum effect, but the author found that the one not corrected for non-continuum effect should be used.

\subsection{The agreement between 15.73 and 15.74 in S\&P text}

To simplify the equations of 15.73 and 15.74, we introduce the following terms.

\begin{equation}
A=\frac{\Delta H_{v} \rho_w}{4 k_a^{'} T_{\infty}}
\end{equation}

\begin{equation}
B=\frac{4 D_{v}^{'} M_{w} P^{0}(T_{\infty})}{\rho_w R T_{\infty}}
\end{equation}

\begin{equation}
C=\frac{\delta H_{v} M_{w}}{R T_{\infty}}
\end{equation}

\begin{equation}
E=\frac{4 M_{w}\sigma_{w}}{R T_{\infty} rho_{w} D_{p}}
\end{equation}

\begin{equation}
F=\frac{6 n_{s} M_{w}}{\pi \rho_{w} (D_{p}^{3}-d_{u}^{3})}
\end{equation}

\begin{equation}
\delta=\frac{\Delta H_{v} \rho_w}{4 k_a^{'} T_{\infty}} D_{p} \frac{d D_{p}}{dt}=A\frac{d D_{p}}{dt}
\end{equation}

Then 15.73 and 15.74 in S\&P can be written respectively as follow,

\begin{equation}
\frac{\delta}{A}=B(S-\frac{1}{1+\delta}\exp(C\frac{\delta}{1+\delta}+E \frac{1}{1+\delta}-F)),
\end{equation}

\begin{equation}
\frac{\delta}{A}=\frac{S-\exp(E-F)}{\frac{1}{B}+A(C-1)},
\end{equation}

where S is the relative humidity. The above 2 equations are solved for $\delta$ as a function of $D_{p}$ and compared. 


Since we are not sure how to reach 15.74 from 15.73, we also use Taylor series expansion to expand 15.73 and truncate the second and above order terms. And we get the following,

\begin{equation}
f(\delta)=\frac{\delta}{A}-B(S-\frac{1}{1+\delta}\exp(C\frac{\delta}{1+\delta}+E \frac{1}{1+\delta}-F))
\end{equation}

\begin{equation}
f(\delta)=f(0)+f^{'}(0)(\delta-0)+\frac{1}{2!}f^{''}(0)(\delta^{2}-0)+...
\end{equation}

Ignoring second and higher order terms, we have
\begin{equation}
\delta=\frac{-f(0)}{f^{'}(0)},
\end{equation}

which can be expressed as
\begin{equation}
\frac{\delta}{A}=\frac{S-\exp(E-F)}{\frac{1}{B}-A \exp(E-F) (1-C+E)}.
\end{equation}

The above three expressions are plotted and compared.

\section{Majeed and Wexler (MW) }

The difference between the SN and MW is that the growth equation in SP is given in terms of diameter rate of change $\frac{d D_{p}}{dt}$, while the MW version is given in terms of mass rate of change, $\frac{dm}{dt}$.

Similar to the derivation given by SP the derivation by MW starts with the mass transfer equation due to diffusion, which is as  (In the following derivation, the subscripts of i which means the size bin i,  are dropped for simplicity.)

\begin{equation}\label{eqn:18}
\frac{dm}{dt}=2 \pi D_{p} D_{v} (c_{w, \infty} - c_{w}^{eq}).  
\end{equation}

By ideal gas law, the above can be expressed as,

\begin{equation}\label{eqn:19}
\frac{dm}{dt}=2 \pi D_{p} D_{v} \frac{M_{w}}{R} (\frac{p_{w, \infty}}{T_{\infty}} - \frac{p_{w}(D_{p}, T_{a})}{T_{a}}).
\end{equation}


\begin{equation}\label{eqn:20}
\frac{dm}{dt}=2 \pi D_{p} D_{v} \frac{M_{w}}{R}  \frac{p^{o}(T_{\infty})} {T_{\infty}} (\frac{p_{w, \infty}}{p^{o}(T_{\infty})} - \frac{p_{w}(D_{p}, T_{a}) T_{\infty}}{p^{o}(T_{\infty}) T_{a}}).
\end{equation}


\begin{equation}\label{eqn:MWgrowth}
\frac{dm}{dt}=2 \pi D_{p} D_{v}  \frac{M_{w}}{R} \frac{p^{o}(T_{\infty})} {T_{\infty}} (\textrm{RH} - \frac{p_{w}(D_{p}, T_{a})}{p^{o}(T_{a})} \frac{p^{o}(T_{a})}{p^{o}(T_{\infty})}\frac{1}{1+\delta}).
\end{equation}

where RH has the same meaning as $S_{v, \infty}$ and $\delta=\frac{\Delta H_{v}}{2 \pi D_{p} k^{'}_{a} T_{\infty}} \frac{dm}{dt}$.
Note that the $\delta$s in SN and MW derivations are different, the former is given in $\frac{dD_{p}}{dt}$ and the latter $\frac{dm}{dt}$.


The next step is to substitute the results from Kelvin (eqn (4) in MW text) and Clausius-Clapeyron equations (eqn (6) in MW text) in the eqn \ref{eqn:MWgrowth}. However, as pointed by James Kelly, the equation (4) in MW text may contain mistake. James suggested that starting from 15.38 of S\&P, we can get 

\begin{equation} 
\ln(\frac{P_{w}(D_{p})}{P^{o}})=\frac{4 M_{w} \sigma_{w}}{R T_{a} \rho_{w}D_{p}}-\frac{M_{w}}{M_{s}}\frac{\nu m_{s}}{\rho_w(\frac{m_p}{\rho_p}-(1-\epsilon)\frac{m_{dry}}{\rho_u})}.
\end{equation}

While equation (4) in M\&W, 

\begin{equation}
\frac{p_{w}(D_{p}, T_{a})}{p^{0}(T_{a}}=\exp(\frac{4 M_{w} \sigma_{w}}{R T_{a} \rho_{w}}-\frac{M_{w}}{M_{s}}\frac{\nu m_{s}}{m_{D}-m_{dry}}),
\end{equation}

where $m_{D}$ in the above equation means droplet mass, same as $m_{p}$ in S\&P. 

Since different text use different symbol to denote the same quantity, the following table is a clarification.


%\begin{table}[htdp]
%\caption{default}
%\begin{center}
%\begin{tabular}{c|c|c}
%--   & S$\&$P  &  M$\&$W
%$m_{s}$  &   solute mass  & solute mass 
%$m_{D}$  & --  &  droplet mass 
%$m_{p}$  & droplet particle  &  dry particle
%\end{tabular}
%\end{center}
%\label{default}
%\end{table}





\begin{equation}\label{eqn:22}
\frac{dm}{dt}=2 \pi D_{p} D_{v}  \frac{M_{w}}{R} \frac{P^{o}(T_{\infty})} {T_{\infty}} (\textrm{RH} - \frac{1}{1+\delta} \exp( \frac{4 M_{w}\sigma_{w}}{R T_{a} \rho_w}\frac{1}{D_{p}}-\frac{ M_{w} \nu m_{s}}{M_{s} (m_{d}-m_{p})}+\frac{\Delta H_{v} M_{w}}{R} (\frac{T_{a}-T_{\infty}}{T_{a} T_{\infty}}))).
\end{equation}

\begin{equation}\label{eqn:23}
\frac{dm}{dt}=2 \pi D_{p} D_{v}  \frac{M_{w}}{R} \frac{P^{o}(T_{\infty})} {T_{\infty}} (\textrm{RH} - \frac{1}{1+\delta} \exp( \frac{4 M_{w}\sigma_{w}}{R T_{a} \rho_w}\frac{1}{D_{p}}-\frac{ M_{w} \nu m_{s}}{M_{s} (m_{d}-m_{p})}+\frac{\Delta H_{v} M_{w}}{R T_{\infty}} (\frac{\delta}{1+\delta}))).
\end{equation}

For $\delta \ll 1$, the above can be simplified as follows,


\begin{equation}\label{eqn:24}
\frac{dm}{dt}=2 \pi D_{p} D_{v}  \frac{M_{w}}{R} \frac{P^{o}(T_{\infty})} {T_{\infty}} (\textrm{RH} - \exp( \frac{4 M_{w}\sigma_{w}}{R T_{a} \rho_w}\frac{1}{D_{p}}-\frac{ M_{w} \nu m_{s}}{M_{s} (m_{d}-m_{p})}) (1+\frac{\Delta H_{v} M_{w}}{R T_{\infty}} \delta)).
\end{equation}


Substituting $\delta$ into \ref{eqn:24}, we get 
\begin{equation}\label{eqn:25}
\frac{dm}{dt}=2 \pi D_{p} D_{v}  \frac{M_{w}}{R} \frac{P^{o}(T_{\infty})} {T_{\infty}} (\textrm{RH} - \exp( \frac{4 M_{w}\sigma_{w}}{R T_{a} \rho_w}\frac{1}{D_{p}}-\frac{ M_{w} \nu m_{s}}{M_{s} (m_{d}-m_{p})}) (1+\frac{\Delta H_{v} M_{w}}{R T_{\infty}} \frac{\Delta H_{v}}{2 \pi D_{p} k^{'}_{a} T_{\infty}} \frac{dm}{dt}).
\end{equation}


Grouping terms containing $\frac{dm}{dt}$, we have  
\begin{eqnarray}\label{eqn:26}
\frac{dm}{dt}(1+\frac{2 \pi D_{p} D_{v} M_{w} P^{o}(T_{\infty})}{R T_{\infty}} \times \\  \nonumber
\exp(\frac{4 M_{w}\sigma_{w}}{R T_{a} \rho_w}\frac{1}{D_{p}}- \frac{ M_{w} \nu m_{s}}{M_{s} (m_{d}-m_{p})})(\frac{\Delta H_{v} M_{w}} {R T_{\infty}}\frac{\Delta H_{v}}{2 \pi D_{p} k^{'}_{a} T_{\infty}})) \\  \nonumber
=2 \pi D_{p} D_{v}  \frac{M_{w}}{R} \frac{P^{o}(T_{\infty})} {T_{\infty}} 
 (\textrm{RH}-\exp( \frac{4 M_{w}\sigma_{w}}{R T_{a} \rho_w} 
\frac{1}{D_{p}}-\frac{ M_{w} \nu m_{s}}{M_{s} (m_{d}-m_{p})}). \nonumber
\end{eqnarray}


\begin{eqnarray}\label{eqn:MWfinal}
\frac{dm}{dt}= (2 \pi D_{p} D_{v}  \frac{M_{w}}{R} \frac{P^{o}(T_{\infty})} {T_{\infty}} 
(\textrm{RH} - \exp( \frac{4 M_{w}\sigma_{w}}{R T_{a} \rho_w} 
\frac{1}{D_{p}}-\frac{ M_{w} \nu m_{s}}{M_{s} (m_{d}-m_{p})})) \\
\times (1+\frac{2 \pi D_{p} D_{v} M_{w} P^{o}(T_{\infty})}{R T_{\infty}} \exp(\frac{4 M_{w}\sigma_{w}}{R T_{a} \rho_w}\frac{1}{D_{p}}-\frac{ M_{w} \nu m_{s}}{M_{s} (m_{d}-m_{p})}) (\frac{\Delta H_{v} M_{w}} {R T_{\infty}}\frac{\Delta H_{v}}{2 \pi D_{p} k^{'}_{a} T_{\infty}}))^{-1}. \nonumber
\end{eqnarray}

The eqn \ref{eqn:MWfinal} agrees with eqn (7) in MW derivation.

\section{James Kelly}

The paper by James Kelly deals with the case when the particle contains insoluble, slightly soluble and  highly soluble components.


\end{document}