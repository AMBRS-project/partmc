\documentclass[12pt]{article}
\usepackage{geometry} % see geometry.pdf on how to lay out the page. There's lots.
\geometry{a4paper} % or letter or a5paper or ... etc
% \geometry{landscape} % rotated page geometry

% See the ``Article customise'' template for come common customisations

\title{Single parameter approach accounting for the solubility and insolubility in an internally mixed aerosol particle}
\author{Joseph Ching}
%\date{} % delete this line to display the current date

%%% BEGIN DOCUMENT
\begin{document}

\maketitle
\tableofcontents

%%%%%%%%%%%%%%%%%%%%%%%%%%%%%%%%

This short report reviews the derivation of the condensational growth equation of a single liquid cloud droplet in three different forms. Section 1 reviews the derivation of the growth equation in terms of the conventional Kelvin's term (curvature term) and Raoult's term (solute term). The equilibrium equation is given by the traditional Kohler's equation. Section 2 contains the derivation in terms of water activity $a_{\rm w}$ based on the results of the section 1. Section 3 details the derivation in terms of hygroscopicity parameters called $\kappa$. 

In the atmosphere, the aerosol particles are observed to exist in complex mixtures. An aerosol particle can contain  soluble materials,  insoluble materials, organic materials and inorganic materials. It is even more complicated that the solubility some materials depends on environmental conditions and the composition of aerosol particle it exists in. 

The purpose of deriving the growth and equilibrium equations in terms of $\kappa$ is to 

\section{Equilibrium equation and Growth equation in traditional framework}

\subsection{Condensational growth equation of a droplet based on S$\&$P}

The derivation is contained in chapter 15 (pp. 801- 803) of Senfeld and Pandis. The following derivations just show the mathematical derivation in more detail.

Assuming the diffusional growth of a cloud droplet is directly proportional to the vapor density gradient and the surface area of the droplet. The derivation starts from the equation of diffusion of mass (eqn15.64 in S$\&$P) which is given as
\begin{equation}\label{eqn:1.1}
\frac{dm}{dt}=2 \pi D_{\rm p} D_{\rm v} (c_{\rm w, \infty} - c_{\rm w}^{eq}),  
\end{equation}
where $D_{\rm p}$ is the droplet wet diameter, $D_{\rm v}$ is the diffusion constant, $c_{\rm w, \infty}$ and $c^{eq}_{\rm w}$ are the environmental vapor density and equilibrium vapor density over the droplet surface respectively.

The total mass of a droplet, $m$ is the sum of the dry mass and water,

\begin{equation}\label{eqn:1.2}
m=\rho_{\rm d}V_{\rm d}+\rho_{\rm w}V_{\rm w}
\end{equation}
 
The rate of change of mass is then expressed as

\begin{equation}\label{eqn:1.3}
\frac{dm}{dt}=\rho_{\rm w} \frac{dV_{\rm w}}{dt}=\rho_{\rm w} \frac{dV_{\rm p}}{dt}  
\end{equation}

To express the growth equation in terms of diameter rate of change, instead of mass rate of change,

\begin{equation}\label{eqn:1.4}
\frac{dm}{dt}=\frac{1}{2} \pi \rho_{\rm w} D_{\rm p}^{2} \frac{dD_{\rm p}}{dt}
\end{equation}

and then put eqn(\ref{eqn:1.4}) into the the eqn(\ref{eqn:1.1}), gives

\begin{equation}\label{eqn:1.5}
\frac{1}{2} \pi \rho_{\rm w} D_{\rm p}^{2} \frac{dD_{\rm p}}{dt}= 2 \pi D_{\rm p} D_{\rm v} (c_{\rm w, \infty} - c_{\rm w}^{\rm eq}),
\end{equation}

and after rearranging we get, 
\begin{equation}\label{eqn:1.6}
D_{\rm p}\frac{dD_{\rm p}}{dt}=4 \frac{D_{\rm v}} {\rho_{\rm w}} (c_{\rm w, \infty} - c_{\rm w}^{eq}).
\end{equation}

Assuming ideal gas law, $P$=$c$$(R / M_{\rm w})$$T$, the eqn(\ref{eqn:1.6}) can be expressed as,

\begin{equation}\label{eqn:1.7}
D_{\rm p}\frac{dD_{\rm p}}{dt}=4 \frac{D_{\rm v}} {\rho_{\rm w}} \frac{M_{\rm w}}{R} \left(\frac{P_{\rm w, \infty}}{T_{\infty}} - \frac{P_{\rm w}(D_{\rm p}, T_{\rm a})}{T_{\rm a}}\right),
\end{equation}
where $P_{\rm w, \infty}$ and $P_{\rm w}(D_{\rm p}, T_{\rm a})$ are the vapor pressures in the environment and on the droplet surface (of diameter $D_{\rm p}$ and temperature $T_{\rm a}$) respectively.

To have the term environmental relative humidity or supersaturation entered into eqn(\ref{eqn:1.7}) as follows,  
\begin{equation}\label{eqn:1.8}
D_{\rm p}\frac{dD_{\rm p}}{dt}=4  \frac{D_{\rm v}} {\rho_{\rm w}} \frac{M_{\rm w}}{R} \frac{P^{\rm 0}(T_{\infty})} {T_{\infty}} \left(\frac{P_{\rm w, \infty}}{P^{\rm 0}(T_{\infty})} - \frac{P_{\rm w}(D_{\rm p}, T_{\rm a}) T_{\infty}}{P^{\rm 0}(T_{\infty}) T_{\rm a}}\right),
\end{equation}
where $P^{\rm 0}(T_{\infty})$ is the saturated vapor pressure in the environment.

To relate the temperature of the environment, $T_{\infty}$ to that on the droplet surface, $T_{\rm a}$, we need to the law of conservation of energy, which implies that the latent heat released during condensation is conducted away from the droplet to the environment. The energy balance is described as follows, 

\begin{equation}\label{eqn:1.9}
2 \pi D_{\rm p} k^{'}_{\rm a} (T_{\infty}-T_{\rm a})= -\Delta H_{\rm v}(\frac{dm}{dt}).
\end{equation}

and can be further expressed in terms of diameter growth, 

\begin{equation}\label{eqn:1.10}
2 \pi D_{\rm p} k^{'}_{\rm a} (T_{\infty}-T_{\rm a})= -\Delta H_{\rm v} \frac{1}{2} \pi \rho_{\rm w} D_{\rm p}^{2} \frac{dD_{\rm p}}{dt},
\end{equation}

and after rearranging we get, 
\begin{equation}\label{eqn:1.11}
T_{\rm a}= T_{\infty} + \frac{\Delta H_{\rm v}  \rho_{\rm w}}{4 k^{'}_{\rm a}} D_{\rm p} \frac{dD_{\rm p}}{dt},
\end{equation}

and write as 
\begin{equation}\label{eqn:1.12}
T_{\rm a}=T_{\infty}(1+\delta),
\end{equation}

where 
\begin{equation}\label{eqn:1.13}
\delta= \frac{\Delta H_{\rm v}  \rho_{\rm w}}{4 k^{'}_{\rm a}T_{\infty}} D_{\rm p} \frac{dD_{\rm p}}{dt}, 
\end{equation}
and $\Delta H_{\rm v}$ is the latent heat of evaporation.


By substituting eqn(\ref{eqn:1.12}) into eqn(\ref{eqn:1.8}) and denote the relative humidity or the supersaturation, $P_{\rm w, \infty}$ / $P^{\rm 0}(T_{\infty})$ as $S_{v, \infty}$, we get 

\begin{equation}\label{eqn:1.14}
D_{\rm p}\frac{dD_{\rm p}}{dt}=4 \frac{D_{\rm v}} {\rho_{\rm w}} \frac{M_{\rm w}}{R} \frac{P^{\rm 0}(T_{\infty})} {T_{\infty}} \left(S_{v, \infty} - \frac{P_{\rm w}(D_{\rm p}, T_{\rm a})}{P^{\rm 0}(T_{\infty})} \frac{1}{1+\delta}\right).
\end{equation}


Write the above equation as follows,
\begin{equation}\label{eqn:1.15}
D_{\rm p}\frac{dD_{\rm p}}{dt}=4 \frac{D_{\rm v}} {\rho_{\rm w}} \frac{M_{\rm w}}{R} \frac{P^{\rm 0}(T_{\infty})} {T_{\infty}} \left(S_{\rm v, \infty} - \frac{P_{\rm w}(D_{\rm p}, T_{\rm a})}{P^{\rm 0}(T_{\rm a})} \frac{P^{\rm 0}(T_{\rm a})}{P^{\rm 0}(T_{\infty})}\frac{1}{1+\delta}\right).
\end{equation}


Then we need to obtain the mathematical relationships for the two ratios of $P_{\rm w}(D_{\rm p}, T_{\rm a})$/$P^{\rm 0}(T_{\rm a})$ and $P^{\rm 0}(T_{\rm a})$ / $P^{\rm 0}(T_{\infty})$. 


For the first ratio, $P_{\rm w}(D_{\rm p}, T_{\rm a})$/$P^{\rm 0}(T_{\rm a})$, it is obtained by the Kohler's equation which is proved and given in S$\&$P eqn(15.38). The equation is 
\begin{equation}\label{eqn:1.16}
 \frac{P_{\rm w}(D_{\rm p}, T_{\rm a})}{P^{\rm 0}(T_{\rm a})}=\exp{\left(\frac{4M_{\rm w} \sigma_{\rm w}}{RT_{\rm a}\rho_{\rm w}D_{\rm p}}-\frac{6n_{\rm ions}M_{\rm w}}{\pi \rho_{\rm w}(D_{\rm p}^{3}-d_{\rm u}^{3})}\right)},
\end{equation} 
where $d_{\rm u}$ is the diameter of the insoluble substances.

The first term in eqn(\ref{eqn:1.16}) is usually called Kelvin or curvature term, the second term is called Raoult or solute term.

The above equation is usually written as 

\begin{equation}\label{eqn:1.17}
\frac{P_{\rm w}(D_{\rm p}, T_{\rm a})}{P^{\rm 0}(T_{\rm a})}=\exp{\left(\frac{A}{D_{\rm p}}-\frac{B}{(D_{\rm p}^{3}-d_{\rm u}^{3})}\right)},
\end{equation} 

with 

\begin{equation}\label{eqn:1.18}
A=\frac{4 M_{\rm w}\sigma_{\rm w}} {R T_{\rm a} \rho_{\rm w}}
\end{equation}

and 

\begin{equation}\label{eqn:1.19}
B=\frac{6 n_{\rm ions} M_{\rm w}}{\pi \rho_{\rm w}}.
\end{equation}

For the second ratio, $P^{\rm 0}(T_{\rm a})$/$P^{\rm 0}(T_{\infty})$, assuming the ratio of the equilibrium vapor pressure at droplet surface temperature, $T_{\rm a}$ to that at the environmental temperature, $T_{\infty}$ is given by the Clausius-Clapeyron equation as follows,

\begin{equation}\label{eqn:1.20}
\frac{P^{\rm 0}(T_{\rm a})}{P^{\rm 0}(T_{\infty})}=\exp \left(\frac{\Delta H_{\rm v} M_{\rm w}}{R} (\frac{T_{\rm a}-T_{\infty}}{T_{\rm a} T_{\infty}})\right).
\end{equation}

which can then be further simplified into 

\begin{equation}\label{eqn:1.21}
\frac{P^{\rm 0}(T_{\rm a})}{P^{\rm 0}(T_{\infty})}=\exp \left(\frac{\Delta H_{\rm v} M_{\rm w}}{R T_{\infty}} \frac{\delta}{1+\delta}\right).
\end{equation}

Substituting the eqn(\ref{eqn:1.16}) and eqn(\ref{eqn:1.21}) into eqn(\ref{eqn:1.15}), we have
the growth equation, 
\begin{equation}\label{eqn:1.22}
D_{\rm p}\frac{dD_{\rm p}}{dt}=4 \frac{D_{\rm v}} {\rho_{\rm w}} \frac{M_{\rm w}}{R} \frac{P^{\rm 0}(T_{\infty})} {T_{\infty}} \left(S_{\rm v, \infty} - \frac{1}{1+\delta} \exp \Big(\frac{4 M_{\rm w}\sigma_{\rm w}}{R T_{\rm a} \rho_{\rm w}}\frac{1}{D_{\rm p}}-\frac{6 n_{\rm ions} M_{\rm w}}{\pi \rho_{\rm w}} \frac{1}{D_{\rm p}^{3}-d_{\rm u}^{3}}+\frac{\Delta H_{\rm v} M_{\rm w}}{R T_{\infty}} \frac{\delta}{1+\delta}\Big)\right).
\end{equation}

Or writing the above equation in terms of  $A$ and $B$, we have

\begin{equation}\label{eqn:1.23}
D_{\rm p}\frac{dD_{\rm p}}{dt}=4 \frac{D_{\rm v}} {\rho_{\rm w}} \frac{M_{\rm w}}{R} \frac{P^{\rm 0}(T_{\infty})} {T_{\infty}} \left(S_{\rm v, \infty} - \frac{1}{1+\delta} \exp\Big(\frac{A}{D_{\rm p}}- \frac{B}{D_{\rm p}^{3}-d_{\rm u}^{3}}+\frac{\Delta H_{\rm v} M_{\rm w}}{R T_{\infty}} \frac{\delta}{1+\delta}\Big)\right).
\end{equation}
%%%%%%%%%%%%%%%%%%%%%%%%%%
\subsection{Equilibrium equation}
To obtain the equilibrium equation, we can put $\delta=0$ into the above equation,  we then have the following,

\begin{equation}\label{eqn:1.24}
S_{\rm v, \infty} = \exp\left(\frac{A}{D_{\rm p}}- \frac{B}{D_{\rm p}^{3}-d_{\rm u}^{3}}\right).
\end{equation}


%%%%%%%%%%%%%%%%%%%%%%%%%%
\subsection{Critical diameter and supersaturation} 

By differentiating eqn(\ref{eqn:1.24}) with respect to droplet diameter $D_{\rm p}$, and solve for $D_{\rm p}$ in the following equation,  

\begin{equation}\label{eqn:1.25}
\frac{\partial S_{\rm v, \infty}}{\partial D_{\rm p}}=0.
\end{equation}

Assuming $B$ does not depends on the wet diameter $D_{\rm p}$ (i.e. assuming Molal osmotic coefficient =1 and the solution is dilute (molarity $m$ is small), we can neglect higher order terms in the Taylor series expansion of eqn(\ref{eqn:3.8}) in section 3.2,  we get
\begin{equation}\label{eqn:1.26}
\frac{A}{D_{\rm p}^{2}}=\frac{3 B D_{\rm p}^{2}}{(D_{\rm p}^{3}- d_{\rm u}^{3})^{2}}.
\end{equation} 

After some algebra, we obtain the following six degree equation.
\begin{equation}\label{eqn:1.27}
A(D_{\rm p}^{*})^{6}-3B(D_{\rm p}^{*})^{4}-2 A(D_{\rm p}^{*})^{3} d_{\rm u}^{3}+ A d_{\rm u}^{6}=0.
\end{equation}

We call the solution to the above equation the critical diameter and denote by $D_{\rm p}^{*}$. If there is no insoluble substance ($d_{\rm u}=0$), the above equation reduces to 

\begin{equation}\label{eqn:1.28}
(D_{\rm p}^{*})^{2}=\frac{3B}{A} 
\end{equation}

and hence 

\begin{equation}\label{eqn:1.29}
D_{\rm p}^{*}=\sqrt{\frac{3B}{A}}= \sqrt {\frac{9}{2 \pi} \frac{n_{\rm ions} RT_{\rm a}}{\sigma_{\rm w}}}.
\end{equation}

This agree with equation 15.30 in S$\&$P.

By substituting the critical diameter into eqn(\ref{eqn:1.24}), we get the corresponding supersaturation, $S^{*}$,

\begin{equation}\label{eqn:1.30}
{\rm ln} S^{*} = \Big(A (\sqrt{\frac{A}{3B}})   -   B (\sqrt{\frac{A}{3B}}) ^{3} \Big).
\end{equation}


\begin{equation}\label{eqn:1.31}
{\rm ln}  S^{*}=(\frac{4A^{3}}{27B})^{1/2}
\end{equation}


%%%%%%%%%%%%%%%%%%%%%%%%%%%%%%%%%
\section{Growth and equilibrium equations as a function of $a_{\rm w}$} 

\subsection{Growth equation as a function of water activity $a_{\rm w}$}

From previously derived result, the growth equation, eqn(\ref{eqn:1.14}), which is given as follow,

\begin{equation}\label{2.1}
D_{\rm p}\frac{dD_{\rm p}}{dt}=4 \frac{D_{\rm v}} {\rho_{\rm w}} \frac{M_{\rm w}}{R} \frac{P^{\rm 0}(T_{\infty})} {T_{\infty}} \left(S_{v, \infty} - \frac{P_{\rm w}(D_{\rm p}, T_{\rm a})}{P^{\rm 0}(T_{\infty})} \frac{1}{1+\delta}\right).
\end{equation}

Write eqn(\ref{eqn:1.14}) as follows,
\begin{equation}\label{eqn:2.2}
D_{\rm p}\frac{dD_{\rm p}}{dt}=4 \frac{D_{\rm v}} {\rho_{\rm w}} \frac{M_{\rm w}}{R} \frac{P^{\rm 0}(T_{\infty})} {T_{\infty}} \left(S_{\rm v, \infty} - \frac{P_{\rm w}(D_{\rm p}, T_{\rm a})}{P^{\rm 0}(T_{\rm a})} \frac{P^{\rm 0}(T_{\rm a})}{P^{\rm 0}(T_{\infty})}\frac{1}{1+\delta}\right).
\end{equation}

By the definition of $a_{\rm w}$, we have

\begin{equation}\label{eqn:2.3}
\frac{P_{\rm w}(D_{\rm p}, T_{\rm a})}{P^{\rm 0}(T_{\rm a})}=a_{\rm w}\exp{(\frac{4 M_{\rm w} \sigma_{\rm w}}{R T_{\rm a} \rho_{\rm w}D_{\rm p}})}.
\end{equation}

And same as section 1.1, assuming the ratio of the equilibrium vapor pressure at droplet surface temperature, $T_{\rm a}$ to that at the environmental temperature, $T_{\infty}$ is given by the Clausius-Clapeyron equation as follows,
  
\begin{equation}\label{eqn:2.4}
\frac{P^{\rm 0}(T_{\rm a})}{P^{\rm 0}(T_{\infty})}=\exp \left(\frac{\Delta H_{\rm v} M_{\rm w}}{R} (\frac{T_{\rm a}-T_{\infty}}{T_{\rm a} T_{\infty}})\right).
\end{equation}

which can be further simplified into 

\begin{equation}\label{eqn:2.5}
\frac{P^{\rm 0}(T_{\rm a})}{P^{\rm 0}(T_{\infty})}=\exp \Big(\frac{\Delta H_{\rm v} M_{\rm w}}{R T_{\infty}} \frac{\delta}{1+\delta}\Big).
\end{equation}

Substituting the eqn(\ref{eqn:2.3}) and eqn(\ref{eqn:2.5}) into eqn(\ref{eqn:2.2}) , we get

\begin{equation}\label{eqn:2.6}
D_{\rm p}\frac{dD_{\rm p}}{dt}=4 \frac{D_{\rm v}} {\rho_{\rm w}} \frac{M_{\rm w}}{R} \frac{P^{\rm 0}(T_{\infty})} {T_{\infty}} \left(S_{\rm v, \infty} - a_{\rm w} \frac{1}{1+\delta} \exp(\frac{4 M_{\rm w}\sigma_{\rm w}}{R T_{\rm a} \rho_{\rm w}}\frac{1}{D_{\rm p}}+\frac{\Delta H_{\rm v} M_{\rm w}}{R T_{\infty}} \frac{\delta}{1+\delta})\right).
\end{equation}

%%%%%%%%%%%%%%%%%%%%%%%%%%%%%%%%%%%%
\subsection{Equilibrium equation as a function of water activity $a_{\rm w}$}

The equilibrium equation is given in S$\&$P 15.18 as follows, 

\begin{equation}\label{eqn:2.7}
S=\frac{P_{\rm w}(D_{\rm p})}{P^{o}}=a_{\rm w}\exp{(\frac{4 M_{\rm w} \sigma_{\rm w}}{R T_{\rm a} \rho_{\rm w} D_{\rm p}})}=\gamma_{\rm w}x_{\rm w}\exp{(\frac{4 M_{\rm w} \sigma_{\rm w}}{R T_{\rm a} \rho_{\rm w} D_{\rm p}})}=\gamma_{\rm w}x_{\rm w}\exp{(\frac{A}{D_{\rm p}})},
\end{equation}

where $\gamma_{\rm w}$ is the water activity coefficient (and $\gamma_{\rm w}$ $\rightarrow$ 1 when the solution is dilute.), and $x_{\rm w}$ is the mole fraction of water which is defined as,

\begin{equation}\label{eqn:2.8}
x_{\rm w}=\frac{n_{\rm w}}{n_{\rm w}+n_{\rm ions}},
\end{equation}

where $n_{\rm w}$ and $n_{\rm ions}$ are the number of moles of water and ions respectively.
The value of $x_{\rm w}$ does not depend on the undissolved part of the soluble material and insoluble material. In other words, it just depends on the number of mole of water molecules and that of ions dissociated when dissolved in water.

By comparing the above growth (eqn(\ref{eqn:2.6})) and equilibrium equations (eqn(\ref{eqn:2.7})) to eqn(\ref{eqn:1.23}) and eqn(\ref{eqn:1.24}), we can see the water activity $a_{w}$ is actually a measure of the solute effect on the single droplet growth and is usually expressed alternatively as the Raoult term in the traditional growth equation, eqn(\ref{eqn:1.23}).

%%%%%%%%%%%%%%%%%%%%%%%%%%%%%%%%%%%%
%\subsection{Critical diameter and supersaturation} 

%We don't derive the critical diameter and supersaturation in terms of water activity $a_{\rm w}$.

%%%%%%%%%%%%%%%%%%%%%%%%%%%%%%%%%%%%
\section{Droplet condensational growth in terms of $\kappa$}

\subsection{Paradigm of the study of effect of the aerosol mixing state on hygroscopicity}

\begin{itemize}

\item Define parameter called hygroscopicity, $\kappa$, in terms of water activity $a_{w}$. Define also parameters, $n_{w}^{a}$ and $V_{w}^{a}$, which are number of mole of water molecules and volume of water associated to aerosol species $a$ respectively. 

\item For inorganic substances only, (e.g. NaCl, ($ \rm NH_{4})_{2} \rm SO_{4}$), derive single droplet growth equation and equilibrium equation in terms $\kappa$.

\item Derive the relationship between the overall $\kappa$ for an aerosol particle (which is a mixture of various species) and the $\kappa$ for an individual species. 

\item For treatment of mixture of organic substances and inorganic substances, assuming the $\kappa$ values for organic substances (given by laboratory measurement) obey the same rule of addition as the inorganic substances do, we can obtain an overall $\kappa$ for the organic-inorganic mixture based on the relationship between the overall and the individual $\kappa$.

\end{itemize}

%%%%%%%%%%%%%%%%%%%%%%%%%%%%%%%%%%%
\subsection{Definition of hygroscopicity $\kappa$}

\begin{equation}\label{eqn:3.1}
a_{\rm w}^{-1}=1+\kappa \frac{V_{\rm s}}{V_{\rm w}}
\end{equation}

\begin{equation}\label{eqn:3.2}
a_{\rm w}=\frac{V_{\rm w}}{V_{\rm w} + \kappa V_{\rm s}}
\end{equation}

where $V_{\rm s}$ is the volume of dry particulate matter, which is the sum of soluble (solute) and insoluble material. And $V_{\rm w}$ is the volume of water.

$a_{\rm w}$ can be alternatively defined as eqn(9.63) and eqn(15.16) in S$\&$P as follows, 
\begin{equation}\label{eqn:3.3}
a_{\rm w}=\gamma_{\rm w}(1+\frac{n_{\rm ions}}{n_{\rm w}})^{-1}
\end{equation}

Equating the above two definitions, we have

\begin{equation}\label{eqn:3.4}
a_{\rm w}=\gamma_{\rm w}(1+\frac{n_{\rm ions}}{n_{\rm w}})^{-1}=(1+\kappa \frac{V_{\rm s}}{V_{\rm w}})^{-1}.
\end{equation}

Then we have a relationship for $\kappa$ as follows,

\begin{equation}\label{eqn:3.5}
\kappa=\left(\frac{(\gamma_{\rm w}-1)n_{\rm_w}+\gamma_{\rm w}n_{\rm ions}}{n_{\rm w}}\right)\frac{V_{\rm w}}{V_{\rm s}}.
\end{equation} 

Assuming $\gamma_{w}$=1, we got 

\begin{equation}\label{eqn:3.6}
\kappa= \frac{V_{\rm w} n_{\rm ions}}{V_{\rm s} n_{\rm w}}.
\end{equation}

This can be further simplified as 

\begin{equation}\label{eqn:3.7}
\kappa= \frac{M_{\rm w}}{V_{\rm s} \rho_{\rm w}}n_{\rm ions}.
\end{equation}

The $n_{\rm ions}$ can be  calculated by eqn(10) in Kreidenweis e al 2005, 

\begin{equation}\label{eqn:3.8}
n_{\rm ions}=\Big((\nu \Phi) + \frac{m M_{\rm w}}{1000} \frac{(\nu \Phi)^{2}}{2!} + \frac{m M_{\rm w}}{1000} \frac{(\nu \Phi)^{3}}{3!}+... \Big) \frac{m_{\rm s}}{M_{\rm s}},
\end{equation}
where $m_{\rm s}$ and $M_{\rm s}$ are the mass and molecular weight of salt respectively. and $\nu$ is the number of ions produced per dissociation of one molecular of the salt. $m$ is molarity of the solution. $\Phi$ is the Molal osmotic coefficient, which depends on the concentration of the solution. It is assumed to be 1 when the solution is dilute. Eqns (\ref{eqn:3.7}) and (\ref{eqn:3.8}) imply that the value of $\kappa$ depends on the amount of water, hence depends on the wet diameter $D_{\rm p}$ and changes as the droplet grows or shrinks.

In PARTMC-MOSAIC., $n_{\rm ions}$ is not calculated by the above formula and instead by MOSAIC.

%%%%%%%%%%%%%%%%%%%%%%%%%%%%%%%%%%
\subsection{Equilibrium equation as function of $\kappa$}

From eqn(\ref{eqn:2.3}) and eqn(\ref{eqn:3.2}),

\begin{equation}\label{eqn:3.9}
S_{\rm v, \infty}=\frac{P_{\rm w}(D_{\rm p})}{P^{o}}=\left(\frac{V_{\rm w}}{V_{\rm w} + \kappa V_{\rm s}}\right)   \exp{\left(\frac{4 M_{\rm w} \sigma_{\rm w}}{R T_{\rm a} \rho_{\rm w} D_{\rm p}}\right)},
\end{equation}

To further express $V_{\rm w}$ in terms of $V_{\rm s}$ and $ D_{\rm p}$, we have the following equilibrium equation in terms of the variable $D_{\rm p}$, 

\begin{equation}\label{eqn:3.11}
S_{\rm v, \infty}=\frac{P_{\rm w}(D_{\rm p})}{P^{o}}=\left(\frac{(\frac{\pi}{6}(D_{\rm p})^{3}-V_{\rm s}) }{(\frac{\pi}{6}(D_{\rm p})^{3}+(\kappa-1)V_{\rm s})}\right) \exp{\left(\frac{4 M_{\rm w} \sigma_{\rm w}}{R T_{\rm a} \rho_{\rm w} D_{\rm p}}\right)}.
\end{equation}


%%%%%%%%%%%%%%%%%%%%%%%%%%%%%%%%%%
\subsection{Growth equation as function of $\kappa$ }

From eqn(\ref{eqn:2.6})

\begin{equation}\label{eqn:3.12}
D_{\rm p}\frac{dD_{\rm p}}{dt}=4 \frac{D_{\rm v}} {\rho_{\rm w}} \frac{M_{\rm w}}{R} \frac{P^{\rm 0}(T_{\infty})} {T_{\infty}} \left(S_{\rm v, \infty} -  \left(\frac{V_{\rm w}}{V_{\rm w} + \kappa V_{\rm s}}\right) \frac{1}{1+\delta} \exp \left(\frac{4 M_{\rm w}\sigma_{\rm w}}{R T_{\rm a} \rho_{\rm w}}\frac{1}{D_{\rm p}}+\frac{\Delta H_{\rm v} M_{\rm w}}{R T_{\infty}} \frac{\delta}{1+\delta}\right)\right).
\end{equation}

To further express $V_{\rm w}$ in terms of $V_{\rm s}$ and $ D_{\rm p}$, we have the following growth equation in terms of the variable $D_{\rm p}$,

\begin{equation}\label{eqn:3.13}
D_{\rm p}\frac{dD_{\rm p}}{dt}=4 \frac{D_{\rm v}}{\rho_{\rm w}}\frac{M_{\rm w}}{R} \frac{P^{\rm 0}(T_{\infty})}{T_{\infty}}\left(S_{\rm v, \infty} - \left(\frac{(\frac{\pi}{6}(D_{\rm p})^{3}-V_{\rm s}) }{(\frac{\pi}{6}(D_{\rm p})^{3}+(\kappa-1)V_{\rm s})}\right) \frac{1}{1+\delta} \exp \left(\frac{4 M_{\rm w}\sigma_{\rm w}}{R T_{\rm a} \rho_{\rm w}} \frac{1}{D_{\rm p}}+\frac{\Delta H_{\rm v} M_{\rm w}}{R T_{\infty}} \frac{\delta}{1+\delta}\right)\right).
\end{equation}


%%%%%%%%%%%%%%%%%%%%%%%%%%%%%%%%%%%
\subsection{Critical diameter and supersaturation in terms of $\kappa$}

By differentiating eqn(\ref{eqn:3.11}) and solving the following equation, we can obtain the cirtical diameter, $D_{\rm p}^{*}$.

\begin{equation}\label{eqn:3.14}
\frac{\partial S_{\rm v, \infty}}{\partial D_{\rm p}}=0
\end{equation}

\begin{equation}\label{eqn:3.15}
\frac{\partial S_{\rm v, \infty}}{\partial D_{\rm p}}= \frac{\partial}{\partial D_{\rm p}} \left( \frac{V_{\rm w}}{V_{\rm w}+\kappa V_{\rm s}}\right) \exp{\left(\frac{A}{D_{\rm p}}\right)}+\exp{\left(\frac{A}{D_{\rm p}}\right)} \left(\frac{-1}{D_{\rm p}^{2}}\right) \left(\frac{V_{\rm w}}{V_{\rm w}+\kappa V_{\rm s}} \right)=0,
\end{equation}

where 
\begin{equation}\label{eqn:3.16}
A=\frac{4M_{\rm w}\sigma_{\rm w}}{RT_{\rm a}\rho_{\rm w}}.
\end{equation}


\begin{equation}\label{eqn:3.17}
\exp{\left(\frac{A}{D_{\rm p}^{*}}\right)} \left( \frac{\kappa V_{\rm s} (\frac{\pi}{2}) (D_{\rm p}^{*})^2-(A\frac{\pi^{2}}{36}(D_{p}^{*})^4+\frac{A\pi}{6}D_{\rm p}^{*}(\kappa-1)V_{\rm s}-\frac{AV_{\rm s}\pi}{6}D_{\rm p}^{*}-\frac{AV_{s}^2 (\kappa-1)}{(D_{\rm p}^{*})^2})} {(\frac{\pi}{6}(D_{\rm p}^{*})^{3}+(\kappa-1)V_{\rm s})^{2}} \right)=0 
\end{equation}

\begin{equation}\label{eqn:3.18}
 \kappa V_{\rm s} (\frac{\pi}{2}) (D_{\rm p}^{*})^2-\left(A\frac{\pi^{2}}{36}(D_{\rm p}^{*})^4+\frac{A\pi}{6}D_{\rm p}^{*}(\kappa-1)V_{\rm s}-\frac{AV_{\rm s}\pi}{6}D_{\rm p}^{*}-\frac{AV_{\rm s}^2 (\kappa-1)}{(D_{\rm p}^{*})^2}\right)=0
\end{equation}

\begin{equation}\label{eqn:3.19}
\frac{A\pi^2}{36}(D_{\rm p}^{*})^{6}-\kappa\frac{\pi}{2}V_{\rm s}(D_{\rm p}^{*})^4+\left(\frac{A\pi \kappa V_{\rm s}}{6}-\frac{A\pi V_{\rm s}}{3}\right)(D_{\rm p}^{*})^{3} -A V_{\rm s}^2(\kappa -1) =0
\end{equation}

The critical diameter $D_{\rm p}^{*}$ can be obtained by solving the above sixth degree equation, assuming the value of $\kappa$ is independent of the droplet diameter. This assumption is reasonable as after the activation of the aerosol particles, the aerosol particles become dilute which make $\kappa$ depends weakly on the volume of the droplet. This can be seen from eqn(\ref{eqn:3.8}) which shows when the solution is dilute, the higher order terms can be neglected and the molal osmotic coefficient approximately equals to 1. $\kappa$  approximately equals to $\nu$, which is a constant.

Comparing eqn(\ref{eqn:3.19}) to eqn(\ref{eqn:1.27}), the two equations are consistent in that they are both in sixth degree. And it is then clear that to get the critical diameter $D_{\rm p}^{*}$ and the corresponding critical supersaturation $S^{*}$, we need to solve a sixth degree polynomial equations, no matter we are using traditional condensational growth equation in terms of $A$ and $B$ or in terms of $\kappa$.  


%%%%%%%%%%%%%%%%%%%%%%%%%%%%%%%%%%%
%\subsection{Critical diameter and supersaturation in Kreidenweis 2007}

%From eqn(\ref{eqn:3.7}), 

%\begin{equation}\label{eqn:3.20}
%\kappa= \frac{M_{\rm w}}{V_{\rm s} \rho_{\rm w}}n_{\rm ions}.
%\end{equation}

%\begin{equation}\label{eqn:3.21}
%\kappa= \frac{M_{\rm w}} {\rho_{\rm w}} \frac{6}{\pi D_{\rm d}^{3}} n_{\rm ions}.
%\end{equation}

%And from eqn(\ref{eqn:3.22}),
%\begin{equation}\label{eqn:25}
%B=\frac{6 n_{\rm ions} M_{\rm w}}{\pi \rho_{\rm w}},
%\end{equation}

%\begin{equation}\label{eqn:3.23}
%B=\kappa D_{\rm d}^{3}
%\end{equation}

%Substituting this into eqn(\ref{eqn:SP23}) and eqn(\ref{eqn:SP25})
%we have 

%\begin{equation}\label{eqn:3.24}
%D_{\rm p}^{*}=\sqrt{\frac{3B}{A}}=\sqrt{\frac{3 \kappa D_{\rm d}^{3}}{A}}
%\end{equation}

%and

%\begin{equation}\label{eqn:3.25}
%{\rm ln}  S^{*}=(\frac{4A^{3}}{27B})^{1/2}=(\frac{4A^{3}}{27\kappa D_{\rm d}^{3} })^{1/2}
%\end{equation}

%or 

%\begin{equation}\label{eqn:3.26}
%\kappa=\frac{4A^{3}} {27D_{\rm d}^{3} {\rm ln}^{2} S^{*} }.
%\end{equation}

%This agrees with equation (10) in Petters and Kreidenweis 2007. However this is an approximate expression for the critical diameter and supersaturation. As mentioned before when carrying out the differentiation of eqn(\ref{eqn:SP18}), we are assuming $B$ is independent of the wet diameter $D_{\rm p}$. We see that $B$ depends on $n_{\rm ions}$ which in turn depends on molarity $m$ and Molal osmotic coefficient $\Phi$ eqn(\ref{eqn:16}). These parameters all change with wet diameter $D_{\rm p}$ as a drop grows or shrinks. 

%%%%%%%%%%%%%%%%%%%%%%%%%%%%%%%%%%%
\section{Summary of the formulae derived}

When deriving the condensational growth equation of a single droplet, we make the following assumptions.
\begin{itemize}
\item Assume the condensational growth of a single liquid water droplet obeys eqn(\ref{eqn:1.1}).
\item Assume the curvature effect on a pure water droplet is the same as that on a solution droplet.
\item Assume the solution droplet is dilute, so that the Raoult term in the Kohler's equation can be approximated as eqn(\ref{eqn:1.19}).
\end{itemize}


\begin{tabular}{c | c | c | c } 
\hline
             &  Traditional  &   In terms of $a_{\rm w}$  &  In terms of $\kappa$ \\
\hline  \hline             
    growth equation  &  Eqn(\ref{eqn:1.23})   &   Eqn(\ref{eqn:2.6})   &    Eqn(\ref{eqn:3.13}) \\ 
\hline 
    equilibrium equation & Eqn(\ref{eqn:1.24})   &  Eqn(\ref{eqn:2.7})  &  Eqn(\ref{eqn:3.11})  \\
\hline    
              & assuming  B is constant &  &  assuming $\kappa$ is constant   \\
               critical diameter $D_{\rm p}^{*}$  &  $6^{\rm th}$ deg eqn.   Eqn(\ref{eqn:1.27}) & not derived here  &independent of $D_{\rm p}$ \\
                                                  & Eqn(\ref{eqn:1.28}) or Eqn(\ref{eqn:1.29})  &                    & $6^{\rm th}$ deg eqn. Eqn(\ref{eqn:3.19})   \\                               
  &  assuming $d_{\rm u}$=0  &    &    \\
                                                
                                               
\hline
 & Eqn(\ref{eqn:1.30}) or Eqn(\ref{eqn:1.31})  &              &                        \\
 critical supersaturation $S^{*}$   &  assuming (i) $d_{\rm u}$=0  &   not derived here  &  not derived here   \\
  & (ii) B is constant  & &  \\
\hline 
\end{tabular}

%%%%%%%%%%%%%%%%%%%%%%%%%%%%%%%%%%%
\section{Algorithm implemented to solve the equilibrium and growth equation}

\subsection{Equilibrium equation as function of $\kappa$}

There is an equilibration subroutine in PARTMC to calculate the equilibrium diameter of the aerosol particle which is in equilibrium with the environmental saturation or supersaturation.
The following equation, $f$, is solved in the equilibration subroutine in PARTMC,


\begin{equation}\label{eqn:5.1}
f(D_{\rm p})=S_{\rm v, \infty}-\left(\frac{V_{\rm w}}{V_{\rm w} + \kappa V_{\rm s}}\right)\exp{\left(\frac{4 M_{\rm w} \sigma_{\rm w}}{R T_{\rm a} \rho_{\rm w} D_{\rm p}}\right)},
\end{equation}

To further express $V_{\rm w}$ in terms of $V_{\rm s}$ and $ D_{\rm p}$, we have the following growth equation in terms of the variable $D_{\rm p}$,

\begin{equation}\label{eqn:5.2}
f(D_{\rm p})=S_{\rm v, \infty}-\left(\frac{\frac{\pi}{6}(D_{\rm p})^{3}-V_{\rm s}} {\frac{\pi}{6}(D_{\rm p})^{3}+(\kappa-1)V_{\rm s}}\right)
\exp{\left(\frac{4 M_{\rm w} \sigma_{\rm w}}{R T_{\rm a} \rho_{\rm w} D_{\rm p}}\right)}
\end{equation}

%%%%%%%%%%%%%%%%%%%%%%%%%%%%%%%%%%%
\subsection{Growth equation as function of $\kappa$}

We introduce the following terms to simplify the eqn(\ref{eqn:3.13})

\begin{equation}\label{eqn:5.3}
A=\frac{\Delta H_{\rm v} \rho_{\rm w}}{4 k_{\rm a}^{'} T_{\infty}}
\end{equation}

\begin{equation}\label{eqn:5.4}
B=\frac{4 D_{\rm v}^{'} M_{\rm w} P^{\rm 0}(T_{\infty})}{\rho_w R T_{\infty}}
\end{equation}

\begin{equation}\label{eqn:5.5}
C=\frac{\Delta H_{\rm v} M_{\rm w}}{R T_{\infty}}
\end{equation}

\begin{equation}\label{eqn:5.6}
E=\frac{4 M_{\rm w}\sigma_{\rm w}}{R T_{\infty} \rho_{\rm w} D_{\rm p}}
\end{equation}

\begin{equation}\label{eqn:5.7}
\delta=\frac{\Delta H_{\rm v} \rho_{\rm w}}{4 k_a^{'} T_{\infty}} D_{\rm p} \frac{d D_{\rm p}}{dt}=A D_{\rm p} \frac{d D_{\rm p}}{dt}
\end{equation}

Then eqn(\ref{eqn:3.12}) can be written in terms of the variables above as follows,

\begin{equation}\label{eqn:5.8}
\frac{\delta}{A}=B \left(S-\left(\frac{\delta}{1+\delta}\right) \left(\frac{V_{\rm w}}{V_{\rm w}+\kappa V_{\rm s}}\right)\exp{\left(C \frac{\delta}{1+\delta} +E \frac{1}{1+\delta}\right)}\right)
\end{equation}

To further express $V_{\rm w}$ in terms of $V_{\rm s}$ and $ D_{\rm p}$, we have the following growth equation in terms of the variable $D_{\rm p}$,

\begin{equation}\label{eqn:5.9}
\frac{\delta}{A}=B \Bigg(S-\left(\frac{\delta}{1+\delta}\right) \left(\frac{\frac{\pi}{6}(D_{\rm p})^{3}-V_{\rm s}} {\frac{\pi}{6}(D_{\rm p})^{3}+(\kappa-1)V_{\rm s}}\right)\exp{\left(C \frac{\delta}{1+\delta} +E \frac{1}{1+\delta}\right)} \Bigg)
\end{equation}

In PARTMC, the following equation, $f$, is the one actually being solved,

\begin{equation}\label{eqn:5.10}
f(\delta)=\frac{\delta}{A}-B \Bigg(S-\left(\frac{\delta}{1+\delta}\right) \left(\frac{\frac{\pi}{6}(D_{\rm p})^{3}-V_{\rm s}} {\frac{\pi}{6}(D_{\rm p})^{3}+(\kappa-1)V_{\rm s}}\right)\exp{\left(C \frac{\delta}{1+\delta} +E \frac{1}{1+\delta}\right)} \Bigg)
\end{equation}
%%%%%%%%%%%%%%%%%%%%%%%%%%%%%%%%%%%
\section{Relationship between $\kappa$ of individual component and overall $\kappa$}

Define $\kappa^{a}$ for individual component as follows, 

\begin{equation}\label{eqn:6.1}
a_{\rm w}^{-1}=1+\kappa^{a} \frac{V_{\rm s}^{a}}{V_{\rm w}^{a}} 
\end{equation}

\begin{equation}\label{eqn:6.2}
a_{w}^{-1}=\gamma_{\rm w}^{-1} (1+\frac{n_{\rm ions}^{a}}{n_{\rm w}^{a}}).
\end{equation}

where
\begin{equation}\label{eqn:6.3}
\sum_{a} n_{\rm ions}^{a}=n_{\rm ions}.
\end{equation}
 and 
 
\begin{equation}\label{eqn:6.4}
\sum_{a} V_{\rm s}^{a}=V_{\rm s}.
\end{equation}

Assuming $\gamma_{w}$=1, we have 

\begin{equation}\label{eqn:6.5}
a_{\rm w}^{-1}=(1+\frac{n_{\rm ions}^{a}}{n_{\rm w}^{a}})
\end{equation}

Without assuming $\gamma_{\rm w}$=1, from eqn(\ref{eqn:6.2}), we have

\begin{equation}\label{eqn:6.6}
n_{\rm w}^{a}=n_{\rm ions}^{a}(\frac{a_{\rm w}^{-1}}{\gamma_{\rm w}^{-1}}-1)^{-1}
\end{equation}

\begin{equation}\label{eqn:6.7}
\sum_{a} n_{\rm w}^{a}=\sum_{a}n_{\rm ions}^{a}(\frac{a_{\rm w}^{-1}}{\gamma_{\rm w}^{-1}}-1)^{-1}
\end{equation}

\begin{equation}\label{eqn:6.8}
\sum_{a} n_{\rm w}^{a}=(\frac{a_{\rm w}^{-1}}{\gamma_{\rm w}^{-1}}-1)^{-1}\sum_{a}n_{\rm ions}^{a}
\end{equation}

\begin{equation}\label{eqn:6.9}
\sum_{a} n_{\rm w}^{a}=(\frac{a_{\rm w}^{-1}}{\gamma_{\rm w}^{-1}}-1)^{-1} n_{\rm ions}
\end{equation}

From eqn(\ref{eqn:3.3}), the RHS of eqn(\ref{eqn:6.9}) is just the $n_{\rm w}$. Then we have 

\begin{equation}\label{eqn:6.10}
\sum_{a} n_{\rm w}^{a}=n_{\rm w}
\end{equation}

The above result states that the total number of mole of water molecules is the sum of the number of moles of water molecules associated to each aerosol species. 

%%%%%%%%%%%%%%%%%%%%%%%%%%%%%%%%%%%%

%From eqn(\ref{eqn:6.1})

%\begin{equation}\label{eqn:6.11}
%V_{\rm w}^{a}=\frac{\kappa^{a} V_{\rm s}^{a}}{a_{\rm w}^{-1}-1}
%\end{equation}

%\begin{equation}\label{eqn:6.12}
%\sum_{a}V_{\rm w}^{a}  =\sum_{a}\frac{\kappa^{a} V_{\rm s}^{a}}{a_{\rm w}^{-1}-1}
%\end{equation}

%\begin{equation}\label{eqn:6.13}
%\sum_{a}V_{\rm w}^{a}  =\frac{\sum_{a}\kappa^{a} V_{\rm s}^{a}}{a_{\rm w}^{-1}-1}
%\end{equation}
%%%%%%%%%%%%%%%%%%%%%%%%%%%%%%%%%%%%%%%

By equating eqn(\ref{eqn:3.1}) to eqn(\ref{eqn:6.1}), we have

\begin{equation}\label{eqn:6.14}
\kappa \frac{V_{\rm s}}{V_{\rm w}}= \kappa^{a} \frac{V_{\rm s}^{a}}{V_{\rm w}^{a}}
\end{equation}

\begin{equation}\label{eqn:6.15}
\sum_{a} \kappa \frac{V_{\rm s}}{V_{\rm w}} V_{\rm w}^{a}= \sum_{a} \kappa^{a} V_{\rm s}^{a}
\end{equation}

\begin{equation}\label{eqn:6.16}
\kappa V_{\rm s}= \sum_{a} \kappa^{a} V_{\rm s}^{a}
\end{equation}

\begin{equation}\label{eqn:6.17}
\kappa= \sum_{a} \kappa^{a} \frac {V_{s}^{a}}{V_{s}} = \sum_{a} \kappa^{a} \epsilon_{a},
\end{equation}
where $\epsilon_{a}$ = $ \frac{V_{s}^{a}}{V_{s}}$ is the volume fraction.

This states that the $\kappa$ for an aerosol particle is the volume weighted sum of individual $\kappa$ value of each component in the aerosol particle.  
%%%%%%%%%%%%%%%%%%%%%%%%%%%%%%%%%%%%%

\section{Appendix I - How to treat insoluble substances in the model PARTMC}

There are two approaches to treat insoluble substances in PARTMC, the first one is include the volume of the insoluble material with the assignment of $\nu^{u}$=$0$ in PARTMC. The second approach is to only include the soluble (solute) part of the aerosol particle. This section is to show that both approaches are mathematically equivalent.
 
Approach 1. 
\begin{equation}\label{eqn:A1}
a_{\rm w}^{-1}=1+\kappa \frac{V_{\rm s}}{V_{\rm w}}=1+ (\sum_{a} \kappa^{a} \epsilon_{a})\frac{V_{\rm s}}{V_{\rm w}}.
\end{equation}

Then,
\begin{equation}\label{eqn:A2}
a_{\rm w}^{-1}=1+ (\kappa^{\rm u} \epsilon_{\rm u} +\kappa^{\rm s'} \epsilon_{\rm s'}) \frac{V_{\rm s}}{V_{\rm w}},
\end{equation}

where
\begin{equation}\label{eqn:A3}
\epsilon_{\rm u}=\frac{V_{\rm u}}{V_{\rm s}}=\frac{V_{\rm u}}{V_{\rm u}+V_{\rm s'}},
\end{equation}
and
\begin{equation}\label{eqn:A4}
\epsilon_{\rm s'}=\frac{V_{\rm s'}}{V_{\rm s}}=\frac{V_{\rm s'}}{V_{\rm u}+V_{\rm s'}}.
\end{equation}

We have,
\begin{equation}\label{eqn:A5}
a_{\rm w}^{-1}=1+ (  \frac{\nu^{\rm u} M_{\rm w} \rho_{\rm u}}{\rho_{\rm w}M_{\rm u}} 
\frac{V_{\rm u}}{V_{\rm s}} +
\frac{\nu^{\rm s'} M_{\rm w} \rho_{\rm s'}}{\rho_{\rm w} M_{\rm s'}}    
\frac{V_{\rm s'}}{V_{\rm s}})
\frac{V_{\rm s}}{V_{\rm w}},
\end{equation}

where $\rm V_{\rm s}$ = $ V_{\rm u}$ + $V_{\rm s'}$. $V_{\rm u}$ and $V_{\rm s'}$ are the volumes of the insoluble and solute respectively.

If in PARTMC, for insoluble substances, the $\nu^{\rm u}$ is prescribed as zero, and hence $\kappa^{\rm u}$, $n^{\rm u}$  are zero from eqn({\ref{eqn:3.7}) and eqn(\ref{eqn:3.8}), we don't need to volume of an insoluble component in an aerosol particle when we use eqn(\ref{eqn:6.17}).

From above, we get the following,

\begin{equation}\label{eqn:A6}
a_{\rm w}^{-1}=1+ \kappa^{\rm s'} \frac{V_{\rm s'}}{V_{\rm w}}.
\end{equation}

Approach 2.

\begin{equation}\label{eqn:A7}
a_{\rm w}^{-1}=1+\kappa \frac{V_{\rm s}}{V_{\rm w}}=1+ (\sum_{a} \kappa^{a} \epsilon_{a})\frac{V_{\rm s}}{V_{\rm w}}.
\end{equation}

Then,

\begin{equation}\label{eqn:A8}
a_{\rm w}^{-1}=1+ \kappa^{\rm s'} \epsilon_{\rm s'} \frac{V_{\rm s}}{V_{\rm w}},
\end{equation}

\begin{equation}\label{eqn:A9}
a_{\rm w}^{-1}=1+ \kappa^{\rm s'} (\frac{V_{\rm s'}}{V_{\rm s}})(\frac{V_{\rm s}}{V_{\rm w}}).
\end{equation}

\begin{equation}\label{eqn:A10}
a_{\rm w}^{-1}=1+ \kappa^{\rm s'} \frac{V_{\rm s'}}{V_{\rm w}}.
\end{equation}

This is the same as approach 1. 

%%%%%%%%%%%%%%%%%%%%%%%%%%%%%%%%%%%%%

\section{Appendix II - The equilibration subroutine in PARTMC}

The equilibration subroutine is to associate the aerosol particle with the required amount of water such that it is in equilibrium with the environmental saturation/supersaturation.

Thus the required amount of water is calculated by the equilibrium equation which is given as follow.

\begin{equation}\label{eqn:A11}
\ln{\Big(\frac{p_{\rm w}(D_{\rm p})}{p^{o}}\Big)}=\frac{4 M_{\rm w} \sigma_{\rm w}}{R T_{\rm a} \rho_{\rm w} D_{\rm p}}-\frac{6 \nu m_{\rm s} M_{\rm w}}{\pi \rho_{\rm w} M_{\rm s}(D_{\rm p}^{3}-d_{\rm u}^{3})}
\end{equation}

\begin{equation}\label{eqn:A12}
f=\ln{\Big(\frac{p_{\rm w}(D_{\rm p})}{p^{o}}\Big)}-\frac{4 M_{\rm w} \sigma_{\rm w}}{R T_{\rm a} \rho_{\rm w} D_{\rm p}}+\frac{6 \nu m_{\rm s} M_{\rm w}}{\pi \rho_{\rm w} M_{\rm s}(D_{\rm p}^{3}-d_{\rm u}^{3})}
\end{equation}

Or in terms of water activity, $a_{\rm w}$, the equilibrium equation is given as (proved previously), 

\begin{equation}\label{eqn:A13}
S=a_{\rm w}\exp{(\frac{4 M_{\rm w} \sigma_{\rm w}}{R T_{\rm a} \rho_{\rm w}D_{\rm p}})}, 
\end{equation}

\begin{equation}\label{eqn:A14}
f=\ln{S}-\ln{a_{\rm w}}+\frac{4 M_{\rm w} \sigma_{\rm w}}{R T_{\rm a} \rho_{\rm w} D_{\rm p}}
\end{equation}

%%%%%%%%%%%%%%%%%%%%%%%%%%%%%%%%%%%%%

\section{Several other relevant formulae from Kreidenweis et al 2005}

Mathematical relationship between the various variables. The supersaturation S can be expressed as follows,

\begin{equation}
S=\frac{p_{\rm w}}{p^{o}(T)}=a_{\rm w} \exp {(\frac{4 \sigma_{\rm sol} \overline{V_{\rm w}}} {RT D_{\rm p} ) } }= \frac {RH}{100},
\end{equation}

where $\sigma_{\rm w}$ is the surface tension of the solution and $\overline V_{\rm w}$ is the partial molar volume of water in the solution which is expressed as, 

\begin{equation}
\overline{V_{\rm w}}=\frac{ M_{\rm w}}{\rho_{\rm sol}} (1 + \frac{x}{\rho_{\rm sol}} \frac{d \rho_{\rm sol}}{dx}).
\end{equation}

$M_{\rm w}$ is the molecular weight of water, $\rho_{\rm sol}$ is the density of the solution, and $x$ is the percentage of solute by weight. Assuming the ideal solution behavior.

\begin{equation}
a_{\rm w}=\gamma_{\rm w} x_{\rm w}
\end{equation}

and alternatively, 

\begin{equation}
\ln {(a_{\rm w})}=\frac{- \nu m M_{\rm w} \Phi }{1000}.
\end{equation}

or 

\begin{equation}\label{eqn:46}
a_{\rm w}^{-1}=\exp{(\frac{\nu m M_{\rm w} \Phi }{1000})},
\end{equation}
$\nu$ is the number of ions of solute present in the solution (after dissociation). 

However in earlier days, 

\begin{equation}
a_{w}^{-1}=1+i \frac{n_{s}}{n_{w}},
\end{equation}

where $i$ is called the van't Hoff factor.

By Taylor series expansion of eqn(\ref{eqn:46}), 

\begin{equation}
a_{w}^{-1}=1+(\frac{\nu m M_{\rm w} \Phi }{1000})+\frac{1}{2}(\frac{\nu m M_{\rm w} \Phi }{1000})^{2}+...
\end{equation}

%%%%%%%%%%%%%%%%%%%%%%%%%%%%%%%%%

\section{Treatment of hygroscopicity, $\nu$ and $\kappa$}  
 
\begin{itemize}

\item An internally mixed aerosol particle could be a mulit-component system containing soluble inorganic salts, soluble organic substances, slightly soluble substances whose dissolution depend the amount of water in the aerosol particle, and also insoluble substances, e.g. soot.

\item When the solute effect is considered regarding the cloud droplet growth, the degree of dissociation of the inorganic salt is one of the parameters needed to be known. 

\item Traditionally, vant' Hoff factor, $i$ is used to quantify the degree of dissociation for inorganic salt. 

\item Some past literatures used, $\nu$ to quantify the number of ions produces per dissociation of one unit of salt. There is correspondence between $i$ and $\nu$ which will be also reviewed in this report.

\item  In the case of NaCl, $\nu$ is 2, $\rm (NH_4)_2SO_4$, $\nu$ is 3.

\item Recent studies show that some of the organic substance existing in the atmosphere can also dissolve in water,

\item Hence, organics contribute to the Raoult term, which describes the solute effect in the and should be included in the Kohler's  framework. 

\item For organics, some studies suggest that $\kappa$ is used to quantify the hygroscopic behavior. 


\item In order to obtain a physical quantity to indicate overall hygroscopicity of an internally mixed aerosol particle, we can either convert $\nu$ of inorganic substances to $\kappa$ or convert $\kappa$ for organic substances to $\nu$. 


\item We choose the former approach. After obtaining a formulation of $\kappa$ for individual inorganic salt, we seek an overall $\kappa$ for a mixture consisting of pure inorganic substances. 

\item The ultimate goal is to obtain an overall $\kappa$ for a general internally mixed aerosol particle, which can contain soluble or insoluble; organic or inorganic substances.

\item Petters and Kreidenweis (2007) propose a single parameter, $\kappa$, representation of hygroscopic growth and condensation nucleus activity.

\item However they did not consider an aerosol particle containing insoluble substances in their theoretical framework.

\item The relationship between the solute in Kohler's equation and $\kappa$ is derived for the cases of 1) no insoluble substance contained 2) some insoluble substances contained.  

\item In Summary, we need to have a formulation for an overall $\kappa$ for a general mixture of soluble and insoluble substances and also organic and inorganic substance. 
 
\end{itemize}

%%%%%%%%%%%%%%%%%%%%%%%%%%%%%%%




\end{document}



