\documentclass[12pt]{article}
\usepackage{geometry} % see geometry.pdf on how to lay out the page. There's lots.
\geometry{a4paper} % or letter or a5paper or ... etc
% \geometry{landscape} % rotated page geometry

% See the ``Article customise'' template for come common customisations

\title{Single parameter approach accounting for the solubility and insolubility in an internally mixed aerosol particle}
\author{Joseph Ching}
%\date{} % delete this line to display the current date

%%% BEGIN DOCUMENT
\begin{document}

\maketitle
\tableofcontents

This short report is about the derivation of the relationship the single parameter proposed by number of previous studies (which account for the hygroscopicity) and the solute term in traditional solute term (Raoult) in Kohler's equation.


\section{Treatment of hygroscopicity, $\nu$ and $\kappa$}  
 
\begin{itemize}

\item An internally mixed aerosol particle could be a mulit-component system containing soluble inorganic salts, soluble organic substances, slightly soluble substances whose dissolution depend the amount of water in the aerosol particle, and also insoluble substances, e.g. soot.

\item When the solute effect is considered regarding the cloud droplet growth, the degree of dissociation of the inorganic salt is one of the parameters needed to be known. 

\item Traditionally, vant' Hoff factor, $i$ is used to quantify the degree of dissociation for inorganic salt. 

\item Some past literatures used, $\nu$ to quantify the number of ions produces per dissociation of one unit of salt. There is correspondence between $i$ and $\nu$ which will be also reviewed in this repot.

\item  In the case of NaCl, $\nu$ is 2, $\rm (NH_4)_2SO_4$, $\nu$ is 3.

\item Recent studies show that some of the organic substance existing in the atmosphere can also dissolve in water,

\item Hence, organics contribute to the Raoult term, which describes the solute effect in the and should be included in the Kohler's  framework. 

\item For organics, some studies suggest that $\kappa$ is used to quantify the hygroscopic behavior. 


\item In order to obtain a physical quantity to indicate overall hygroscopicity of an internally mixed aerosol particle, we can either convert $\nu$ of inorganic substances to $\kappa$ or convert $\kappa$ for organic substances to $\nu$. 


\item We choose the former approach. After obtaining a formulation of $\kappa$ for individual inorganic salt, we seek an overall $\kappa$ for a mixture consisting of pure inorganic substances. 

\item The ultimate goal is to obtain an overall $\kappa$ for a general internally mixed aerosol particle, which can contain soluble or insoluble; organic or inorganic substances.

\item Petters and Kreidenweis (2007) propose a single parameter, $\kappa$, representation of hygroscopic growth and condensation nucleus activity.

\item However they did not consider an aerosol particle containing insoluble substances in their theoretical framework.

\item The relationship between the solute in Kohler's equation and $\kappa$ is derived for the cases of 1) no insoluble substance contained 2) some insoluble substances contained.  

\item In Summary, we need to have a formulation for an overall $\kappa$ for a general mixture of soluble and insoluble substances and also organic and inorganic substance. 
 
\subsection{Petters and Kreidenweis (2007) $\kappa$ formulation (for aerosol particles contain no insoluble substances)}

Saturation vapor pressure ratio, $S$ over a solution droplet is given by 

\begin{equation}\label{eqn:83}
S=a_w \exp (\frac{4 \sigma_{s/a} M_{w}} {RT\rho_wD}). 
\end{equation}

$\kappa$ is by the following definition, defined through $a_w$,

\begin{equation}\label{eqn:84}
\frac{1}{a_w}=1+\kappa \frac{V_s}{V_w},
\end{equation}  
where $ V_s$ and $ V_w$ are volume of dry particulate (dry salt) and water respectively.

Assuming ZSR assumption applies, i.e.  $ V_w = \sum_{i} V_{wi}$, where $V_{wi}$ is the water content associated with each component in the internally mixed aerosol particle, and also $a_{wi}$ = $ a_{w}$. 

Then rearranging the above equation, we have

\begin{equation}\label{eqn:85}
V_w = \frac {a_{w}}{(1-a_{w})} \kappa V_{s}. 
\end{equation} 
 
Also write eqn \ref{eqn:84} for each component in the internally mixed aerosol particle, 

\begin{equation}\label{eqn:86}
 \frac{1}{a_{wi}}=1+\kappa_{i} \frac{V_{si}}{V_{wi}}.
\end{equation} 

Then similarly, we have 

\begin{equation}\label{eqn:87}
V_{wi} = \frac {a_{wi}}{(1-a_{wi})} \kappa_{i} V_{si}
\end{equation}

Notice that $ V_w = \sum_{i} V_{wi}$, then we have

\begin{equation}\label{eqn:88}
V_{w} = \sum_{i} V_{wi} = \sum_{i} \frac {a_{wi}}{(1-a_{wi})} \kappa_{i} V_{si}.
\end{equation}

Since $a_{wi}$=$a_{w}$, we have
\begin{equation}\label{eqn:89}
V_{w}  = \frac {a_{w}}{(1-a_{w})} \sum_{i} \kappa_{i} V_{si},
\end{equation}
which is equation (3) in Petters and Kreidenweis (2007).

Recognizing the fact that $V_{T}$=$V_{s}+V_{w}$, where $V_{T}$ is the total volume of the system, water plus solute, when there is no insoluble substance and rearranging eqn \ref{eqn:89}, we then have

\begin{equation}
(V_{T}-V_{s}) (1-a_{w})=a_{w}\sum_{i} \kappa_{i} V_{si}.
\end{equation}


\begin{equation}
V_{T}-V_{s} =a_{w}\Big( \sum_{i} \kappa_{i} V_{si} +(V_{T}-V_{s}) \Big)
\end{equation}


\begin{equation}
a_{w}= \frac {V_{T}-V_{s} }{\sum_{i} \kappa_{i} V_{si} +(V_{T}-V_{s})}
\end{equation}


\begin{equation}
a_{w}= \frac {V_{T}-V_{s} }{V_{s} \sum_{i} \kappa_{i} \frac{V_{si}}{V_{s}} +(V_{T}-V_{s})}
\end{equation}


\begin{equation}
a_{w}= \frac {V_{T}-V_{s} }{V_{T} - V_{s} (1- \sum_{i} \kappa_{i} \epsilon_{i})}
\end{equation}

 or
 
 \begin{equation}
a_{w}= \frac {D^{3}-D_{d}^{3} }{D^{3}-D_{d}^{3} (1- \sum_{i} \kappa_{i} \epsilon_{i})},
\end{equation}

where 
$D$ and $D_{d}$ are the diameter of the droplet and the dry salt respectively and $\epsilon_{i}$ is the volume fraction of component $i$.

\begin{equation}
a_{w}= \frac {D^{3}-D_{d}^{3} }{D^{3}-D_{d}^{3} (1- \kappa)},
\end{equation}

where $\kappa$ = $\sum_{i} \kappa_{i} \epsilon_{i}$.

Substitute above into eqn(\ref{eqn:83}), we get the following

\begin{equation}\label{eqn:99}
S=\frac {D^{3}-D_{d}^{3} }{D^{3}-D_{d}^{3} (1- \kappa)}\exp (\frac{4 \sigma_{s/a} M_{w}} {RT\rho_wD}).
\end{equation}

To obtain the relationship between $\kappa$ and the solute term in Kohler's equation, we notice from equation 15.23 in S$\&$P, we have 

\begin{equation}
\ln \Big(\frac{P_{w}(D_{p})}{p^{o}} \Big)=\frac{4M_{w}\sigma_{w}}{RT\rho_{w}D_{p}}-\ln(\gamma_{w})-\ln(1+\frac{6 n_{s} {\overline{v_{w} } } } {\pi D_{p}^{3} } ).
\end{equation}

Assuming $\gamma_{w}$, the activity coefficient (refer to p.783 S$\&$P), equals 1, noting that $\ln(\frac{P_{w}((D_{p})}{p^{o}})$=$\ln(S)$  and $\ln(1+x)$$\approx x$  , we have

\begin{equation}
\ln \Big(\frac{P_{w}(D_{p})}{p^{o}} \Big)=\ln(S)=\frac{4M_{w}\sigma_{w}}{RT\rho_{w}D_{p}}-\frac{6 n_{s} {\overline{v_{w} } } } {\pi D_{p}^{3} }.  
\end{equation}

Then 

\begin{equation}
S=\exp{(\frac{4M_{w}\sigma_{w}}{RT\rho_{w}D_{p}})} \exp{( - \frac{6 n_{s} {\overline{v_{w} } } } {\pi D_{p}^{3}})},  
\end{equation}

comparing the above to eqn(\ref{eqn:99}), we can see

\begin{equation}
a_{w}=\exp{(- \frac {6 n_{s} {\overline{v_{w} } } } {\pi D_{p}^{3}}) }
\end{equation}

and as above assuming again $\frac{6 n_{s} \overline{v_{w}} } {\pi D_{p}^{3}}$ $\ll$ 1,

\begin{equation}
\frac{1}{a_{w}}=1+\frac{6 n_{s} \overline{v_{w}} } {\pi D_{p}^{3}} 
\end{equation}

\begin{equation}
1+\kappa \frac{V_{s}}{V_{w}}=1+\frac{6 n_{s} \overline{v_{w}} } {\pi D_{p}^{3}} 
\end{equation}

Assuming dilute solution, $\overline{v_{w}}$ $\approx$ $\frac{V_{w}}{n_{w}}$, where $\overline{v_{w}}$ is molar partial volume of water in the solution, 

\begin{equation}\label{eqn:106}
\kappa=\frac{n_{s} V_{w}}{ n_{w}} \frac{1} {V_{p}} \frac{V_{w}}{V_{s}}
\end{equation}

\begin{equation}\label{eqn:107}
\kappa=\frac{n_{s}V_{w}^{2}}{n_{w}V_{p} V_{s}}.
\end{equation}

If $V_{w}$$\gg$ $V_{s}$, the above can be approximated as,

\begin{equation}\label{eqn:108}
\kappa=\frac{n_{s}V_{w}^{2}}{n_{w}V_{p} V_{s}}= \frac{n_{s}V_{w}^{2}}{n_{w} (V_{w}+V_{s}) V_{s}}
\end{equation}

\begin{equation}\label{eqn:109}
\kappa=\frac{n_{s}V_{w}^{2}}{n_{w} V_{w}V_{s}}
\end{equation}

\begin{equation}\label{eqn:110}
\kappa=\frac{n_{s}V_{w}}{n_{w} V_{s}}=\frac{M_{w} \rho_{s}}{ M_{s} \rho_{w} }
\end{equation}


If $V_{s}$$\gg$ $V_{w}$, the above can be approximated as,


\begin{equation}\label{eqn:111}
\kappa=\frac{n_{s}V_{w}^{2}}{n_{w}V_{p} V_{s}}= \frac{n_{s}V_{w}^{2}}{n_{w} (V_{w}+V_{s}) V_{s}}
\end{equation}

\begin{equation}\label{eqn:112}
\kappa=\frac{n_{s}V_{w}^{2}}{n_{w} V_{s}^{2}}= (\frac{M_{w} \rho_{s}}{ M_{s} \rho_{w}})^{2} \frac{n_{w}}{n_{s}}
\end{equation}.

From eqn \ref{eqn:110}, when the solution droplet (aerosol particle)  is dilute, the value of $\kappa$ does not depends on the amount of salt in the aerosol particle. However, when the solution is not dilute, the value of $\kappa$ depends on the number of moles of water and salt (solute)  (i.e. $n_{w}$ and $n_{s}$) in the solution.

\subsection{$\kappa$ formulation for aerosol particles contain insoluble}

We are still not sure about whether the definition of water activity, $a_{w}$ should be revised when the solution contain insoluble substances. Therefore we still use the same definition of $a_{w}$ to derive the relationship between $\kappa$ and the solute term in the Kohler equation.


Equation (5) in Petters and  Kreidenweis (2007) or eqn(\ref{eqn:89}) above  should be rewritten as follows to account for insoluble substances, since $V_{p}$=$V_{s}$+$V_{w}$+$V_{u}$,

\begin{equation}
V_{p}-V_{s}-V_{u}=\frac {a_{w}}{(1-a_{w})} \sum_{i} \kappa_{i} V_{si},
\end{equation}
where $V_{p}$, $V_{w}$ and $V_{u}$ are the volumes of the whole droplet, the water and the insoluble substances in the aerosol particle respectively.


\begin{equation}
a_{w}= \frac {V_{p}-V_{s}-V_{u}}{V_{p} -V_{u}- V_{s} (1- \sum_{i} \kappa_{i} \epsilon_{i})}
\end{equation}

 or
 
 \begin{equation}
a_{w}= \frac {D^{3}-D_{d}^{3}-d_{u}^{3}}{D^{3}- d_{u}^{3}- D_{d}^{3} (1- \sum_{i} \kappa_{i} \epsilon_{i})},
\end{equation}

where 
$D$, $d_{u}$ and $D_{d}$ are the diameter of the droplet, the diameter of insoluble substance (assuming spherical shape of insoluble substances) and the dry salt respectively and $\epsilon_{i}$ is the volume fraction of component $i$ respectively.

\begin{equation}
a_{w}= \frac {D^{3}-D_{d}^{3}-d_{u}^{3} }{D^{3}-d_{u}^{3}-D_{d}^{3} (1- \kappa)},
\end{equation}

where $\kappa$ = $\sum_{i} \kappa_{i} \epsilon_{i}$.


Following similar strategy as we do for aerosol particles contain no insoluble substances, we have

\begin{equation}
S(D)= \frac {D^{3}-D_{d}^{3}-d_{u}^{3} }{D^{3}-d_{u}^{3}-D_{d}^{3} (1- \kappa)}\exp (\frac{4 \sigma_{s/a} M_{w}} {RT\rho_wD}). 
\end{equation}


To obtain the relationship between $\kappa$ and the solute term in Kohler's equation, we notice from equation 15.37 in S$\&$P, we have 

\begin{equation}
\ln \Big(\frac{P_{w}(D_{p})}{p^{o}} \Big)=\frac{4M_{w}\sigma_{w}}{RT\rho_{w}D_{p}}-\ln(\gamma_{w})-\ln \Big(1+\frac{n_{s} {\overline{v_{w} } } } {(\pi/6) (D_{p}^{3}-d_{u}^{3}) - n_{s} \overline{v_{s}} } \Big).
\end{equation}

Assuming $\gamma_{w}$, the activity coefficient (refer to p.783 S$\&$P), equals 1, noting that $\ln(\frac{P_{w}((D_{p})}{p^{o}})$=$\ln(S)$  and $\ln(1+x)$$\approx x$  , we have

\begin{equation}
\ln \Big(\frac{P_{w}(D_{p})}{p^{o}} \Big)=\ln(S)=\frac{4M_{w}\sigma_{w}}{RT\rho_{w}D_{p}}-\frac{n_{s} {\overline{v_{w} } } } {(\pi/6) (D_{p}^{3}-d_{u}^{3}) - n_{s} \overline{v_{s}} } . 
\end{equation}

Furthering assuming it is a very dilute solution, 

\begin{equation}
\ln \Big(\frac{P_{w}(D_{p})}{p^{o}} \Big)=\ln(S)=\frac{4M_{w}\sigma_{w}}{RT\rho_{w}D_{p}}-\frac{n_{s} {\overline{v_{w} } } } {(\pi/6) (D_{p}^{3}-d_{u}^{3})} . 
\end{equation}

\begin{equation}
S=\exp{(\frac{4M_{w}\sigma_{w}}{RT\rho_{w}D_{p}})} \exp{\Big( - \frac{6 n_{s} {\overline{v_{w} } } } {\pi (D_{p}^{3} -d_{u}^{3})}\Big)},  
\end{equation}

comparing the above to eqn(\ref{eqn:99}), we can see

\begin{equation}
a_{w}=\exp{\Big(- \frac {6 n_{s} {\overline{v_{w} } } } {\pi (D_{p}^{3}-d_{u}^{3})} \Big)}
\end{equation}

and as above assuming again $\frac{6 n_{s} \overline{v_{w}} } {\pi D_{p}^{3}}$ $\ll$ 1,

\begin{equation}
\frac{1}{a_{w}}=1+\frac{6 n_{s} \overline{v_{w}} } {\pi (D_{p}^{3}-d_{u}^{3})} 
\end{equation}

\begin{equation}\label{eqn:41}
1+\kappa \frac{V_{s}}{V_{w}}=1+\frac{6 n_{s} \overline{v_{w}} } {\pi (D_{p}^{3}-d_{u}^{3})} 
\end{equation}

Assuming dilute solution, $\overline{v_{w}}$ $\approx$ $\frac{V_{w}}{n_{w}}$, where $\overline{v_{w}}$ is molar partial volume of water in the solution, 


\begin{equation}\label{eqn:42}
\kappa=\frac{n_{s}V_{w}^{2}}{n_{w}(V_{p}-V_{u}) V_{s}}.
\end{equation}

or 

\begin{equation}\label{eqn:42}
\kappa=\frac{n_{s}V_{w}^{2}}{n_{w}(V_{w}+V_{s}) V_{s}}.
\end{equation}



\subsection{the relationship between $i$ and $\nu$}

\subsection{few concepts needed to be clarify}


\item $i$ and $\nu$
\item water activity, $a_{w}$
\item water activity coefficient $\gamma_{w}$
\item partial molar volume of water $\overline{v_{w}}$
\item need to find out about whether $a_{w}$ change when the solution contain insoluble or slightly soluble substances.

\end{itemize}









\end{document}



